\documentclass[11pt]{article}
\usepackage[english]{babel}
\usepackage[utf8]{inputenc}
\usepackage{graphicx}
\title{Emacs}
\author{Raúl Montes}
\begin{document}
\begin{titlepage}
\newcommand{\HRule}{\rule{\linewidth}{0.5mm}}
\center
\textsc{\LARGE Universidad de Sonora}\\[1.5cm]
\HRule \\[0.4cm]
{ \huge \bfseries Primera vez en Emacs}\\[0.4cm] 
\HRule \\[1.1cm]
\begin{minipage}{0.4\textwidth}
\begin{flushleft} \large
\emph{Autor:}\\
Raúl \textsc{Montes} 
\end{flushleft}
\end{minipage}
\begin{minipage}{0.4\textwidth}
\begin{flushright} \large
\emph{Profesor:} \\
Dr. Carlos \textsc{Lizárraga} 
\end{flushright}
\end{minipage}\\[1.5cm]
{\date\ 7 de febrero de 2017\large }\\[2cm] 
\includegraphics[height=8cm]{logounison.png}\\[1cm] 
\end{titlepage}
\begin{enumerate}
\item ¿Cual es tu primera impresión del uso de bash/Emacs?
\\
-Me parece un editor de texto muy útil y fácil de usar.
\item ¿Ya lo habías utilizado?
\\
Si, lo había usado para hacer códigos en Fortran.
\item ¿Qué cosas se te dificultaron más en bash/Emacs?
\\  
-Hasta ahora no me he topado con dificultades porque solo he usado lo básico.
\item ¿Qué ventajas les ves a Emacs?
\\
-Que puedes escoger el formato de código y te muestra con colores que es cada cosa y tus errores.

\item ¿Qué es lo que más te llamó la atención en el desarrollo de esta actividad?
\\
-La descarga de gran cantidad de datos a la computadora usando solo la terminal, no sabía que había maneras de hacer eso.
\item ¿Qué cambiarías en esta actividad?
\\
-Tal vez que los datos fuesen más variantes para tener que filtrar más cosas, en mi caso la estación meteorológica que utilicé no tenía muchos datos disponibles.
\item ¿Que consideras que falta en esta actividad?
\\
- mas manejo de gráficas, son muy útiles en este tipo de cosas.
\item ¿Puedes compartir alguna referencia nueva que consideras útil y no se haya contemplado?
\\
- Me sirvió lo que ya estaba disponible en la plataforma.
\item ¿Algún comentario adicional que desees compartir?
Me gustó mucho esta práctica pero me gustaría entenderlo mejor porque también se me hizo muy confusa, siempre batallo para usar comandos.
\end{enumerate}
    
    
    
    
    
    
    
   

\end{document}
